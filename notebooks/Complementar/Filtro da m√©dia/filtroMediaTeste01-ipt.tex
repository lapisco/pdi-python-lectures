
% Default to the notebook output style

    


% Inherit from the specified cell style.




    
\documentclass[11pt]{article}

    
    
    \usepackage[T1]{fontenc}
    % Nicer default font than Computer Modern for most use cases
    \usepackage{palatino}

    % Basic figure setup, for now with no caption control since it's done
    % automatically by Pandoc (which extracts ![](path) syntax from Markdown).
    \usepackage{graphicx}
    % We will generate all images so they have a width \maxwidth. This means
    % that they will get their normal width if they fit onto the page, but
    % are scaled down if they would overflow the margins.
    \makeatletter
    \def\maxwidth{\ifdim\Gin@nat@width>\linewidth\linewidth
    \else\Gin@nat@width\fi}
    \makeatother
    \let\Oldincludegraphics\includegraphics
    % Set max figure width to be 80% of text width, for now hardcoded.
    \renewcommand{\includegraphics}[1]{\Oldincludegraphics[width=.8\maxwidth]{#1}}
    % Ensure that by default, figures have no caption (until we provide a
    % proper Figure object with a Caption API and a way to capture that
    % in the conversion process - todo).
    \usepackage{caption}
    \DeclareCaptionLabelFormat{nolabel}{}
    \captionsetup{labelformat=nolabel}

    \usepackage{adjustbox} % Used to constrain images to a maximum size 
    \usepackage{xcolor} % Allow colors to be defined
    \usepackage{enumerate} % Needed for markdown enumerations to work
    \usepackage{geometry} % Used to adjust the document margins
    \usepackage{amsmath} % Equations
    \usepackage{amssymb} % Equations
    \usepackage{textcomp} % defines textquotesingle
    % Hack from http://tex.stackexchange.com/a/47451/13684:
    \AtBeginDocument{%
        \def\PYZsq{\textquotesingle}% Upright quotes in Pygmentized code
    }
    \usepackage{upquote} % Upright quotes for verbatim code
    \usepackage{eurosym} % defines \euro
    \usepackage[mathletters]{ucs} % Extended unicode (utf-8) support
    \usepackage[utf8x]{inputenc} % Allow utf-8 characters in the tex document
    \usepackage{fancyvrb} % verbatim replacement that allows latex
    \usepackage{grffile} % extends the file name processing of package graphics 
                         % to support a larger range 
    % The hyperref package gives us a pdf with properly built
    % internal navigation ('pdf bookmarks' for the table of contents,
    % internal cross-reference links, web links for URLs, etc.)
    \usepackage{hyperref}
    \usepackage{longtable} % longtable support required by pandoc >1.10
    \usepackage{booktabs}  % table support for pandoc > 1.12.2
    \usepackage[normalem]{ulem} % ulem is needed to support strikethroughs (\sout)
                                % normalem makes italics be italics, not underlines
    

    
    
    % Colors for the hyperref package
    \definecolor{urlcolor}{rgb}{0,.145,.698}
    \definecolor{linkcolor}{rgb}{.71,0.21,0.01}
    \definecolor{citecolor}{rgb}{.12,.54,.11}

    % ANSI colors
    \definecolor{ansi-black}{HTML}{3E424D}
    \definecolor{ansi-black-intense}{HTML}{282C36}
    \definecolor{ansi-red}{HTML}{E75C58}
    \definecolor{ansi-red-intense}{HTML}{B22B31}
    \definecolor{ansi-green}{HTML}{00A250}
    \definecolor{ansi-green-intense}{HTML}{007427}
    \definecolor{ansi-yellow}{HTML}{DDB62B}
    \definecolor{ansi-yellow-intense}{HTML}{B27D12}
    \definecolor{ansi-blue}{HTML}{208FFB}
    \definecolor{ansi-blue-intense}{HTML}{0065CA}
    \definecolor{ansi-magenta}{HTML}{D160C4}
    \definecolor{ansi-magenta-intense}{HTML}{A03196}
    \definecolor{ansi-cyan}{HTML}{60C6C8}
    \definecolor{ansi-cyan-intense}{HTML}{258F8F}
    \definecolor{ansi-white}{HTML}{C5C1B4}
    \definecolor{ansi-white-intense}{HTML}{A1A6B2}

    % commands and environments needed by pandoc snippets
    % extracted from the output of `pandoc -s`
    \providecommand{\tightlist}{%
      \setlength{\itemsep}{0pt}\setlength{\parskip}{0pt}}
    \DefineVerbatimEnvironment{Highlighting}{Verbatim}{commandchars=\\\{\}}
    % Add ',fontsize=\small' for more characters per line
    \newenvironment{Shaded}{}{}
    \newcommand{\KeywordTok}[1]{\textcolor[rgb]{0.00,0.44,0.13}{\textbf{{#1}}}}
    \newcommand{\DataTypeTok}[1]{\textcolor[rgb]{0.56,0.13,0.00}{{#1}}}
    \newcommand{\DecValTok}[1]{\textcolor[rgb]{0.25,0.63,0.44}{{#1}}}
    \newcommand{\BaseNTok}[1]{\textcolor[rgb]{0.25,0.63,0.44}{{#1}}}
    \newcommand{\FloatTok}[1]{\textcolor[rgb]{0.25,0.63,0.44}{{#1}}}
    \newcommand{\CharTok}[1]{\textcolor[rgb]{0.25,0.44,0.63}{{#1}}}
    \newcommand{\StringTok}[1]{\textcolor[rgb]{0.25,0.44,0.63}{{#1}}}
    \newcommand{\CommentTok}[1]{\textcolor[rgb]{0.38,0.63,0.69}{\textit{{#1}}}}
    \newcommand{\OtherTok}[1]{\textcolor[rgb]{0.00,0.44,0.13}{{#1}}}
    \newcommand{\AlertTok}[1]{\textcolor[rgb]{1.00,0.00,0.00}{\textbf{{#1}}}}
    \newcommand{\FunctionTok}[1]{\textcolor[rgb]{0.02,0.16,0.49}{{#1}}}
    \newcommand{\RegionMarkerTok}[1]{{#1}}
    \newcommand{\ErrorTok}[1]{\textcolor[rgb]{1.00,0.00,0.00}{\textbf{{#1}}}}
    \newcommand{\NormalTok}[1]{{#1}}
    
    % Additional commands for more recent versions of Pandoc
    \newcommand{\ConstantTok}[1]{\textcolor[rgb]{0.53,0.00,0.00}{{#1}}}
    \newcommand{\SpecialCharTok}[1]{\textcolor[rgb]{0.25,0.44,0.63}{{#1}}}
    \newcommand{\VerbatimStringTok}[1]{\textcolor[rgb]{0.25,0.44,0.63}{{#1}}}
    \newcommand{\SpecialStringTok}[1]{\textcolor[rgb]{0.73,0.40,0.53}{{#1}}}
    \newcommand{\ImportTok}[1]{{#1}}
    \newcommand{\DocumentationTok}[1]{\textcolor[rgb]{0.73,0.13,0.13}{\textit{{#1}}}}
    \newcommand{\AnnotationTok}[1]{\textcolor[rgb]{0.38,0.63,0.69}{\textbf{\textit{{#1}}}}}
    \newcommand{\CommentVarTok}[1]{\textcolor[rgb]{0.38,0.63,0.69}{\textbf{\textit{{#1}}}}}
    \newcommand{\VariableTok}[1]{\textcolor[rgb]{0.10,0.09,0.49}{{#1}}}
    \newcommand{\ControlFlowTok}[1]{\textcolor[rgb]{0.00,0.44,0.13}{\textbf{{#1}}}}
    \newcommand{\OperatorTok}[1]{\textcolor[rgb]{0.40,0.40,0.40}{{#1}}}
    \newcommand{\BuiltInTok}[1]{{#1}}
    \newcommand{\ExtensionTok}[1]{{#1}}
    \newcommand{\PreprocessorTok}[1]{\textcolor[rgb]{0.74,0.48,0.00}{{#1}}}
    \newcommand{\AttributeTok}[1]{\textcolor[rgb]{0.49,0.56,0.16}{{#1}}}
    \newcommand{\InformationTok}[1]{\textcolor[rgb]{0.38,0.63,0.69}{\textbf{\textit{{#1}}}}}
    \newcommand{\WarningTok}[1]{\textcolor[rgb]{0.38,0.63,0.69}{\textbf{\textit{{#1}}}}}
    
    
    % Define a nice break command that doesn't care if a line doesn't already
    % exist.
    \def\br{\hspace*{\fill} \\* }
    % Math Jax compatability definitions
    \def\gt{>}
    \def\lt{<}
    % Document parameters
    \title{filtroMediaTeste01-ipt}
    
    
    

    % Pygments definitions
    
\makeatletter
\def\PY@reset{\let\PY@it=\relax \let\PY@bf=\relax%
    \let\PY@ul=\relax \let\PY@tc=\relax%
    \let\PY@bc=\relax \let\PY@ff=\relax}
\def\PY@tok#1{\csname PY@tok@#1\endcsname}
\def\PY@toks#1+{\ifx\relax#1\empty\else%
    \PY@tok{#1}\expandafter\PY@toks\fi}
\def\PY@do#1{\PY@bc{\PY@tc{\PY@ul{%
    \PY@it{\PY@bf{\PY@ff{#1}}}}}}}
\def\PY#1#2{\PY@reset\PY@toks#1+\relax+\PY@do{#2}}

\expandafter\def\csname PY@tok@ge\endcsname{\let\PY@it=\textit}
\expandafter\def\csname PY@tok@bp\endcsname{\def\PY@tc##1{\textcolor[rgb]{0.00,0.50,0.00}{##1}}}
\expandafter\def\csname PY@tok@gs\endcsname{\let\PY@bf=\textbf}
\expandafter\def\csname PY@tok@nf\endcsname{\def\PY@tc##1{\textcolor[rgb]{0.00,0.00,1.00}{##1}}}
\expandafter\def\csname PY@tok@ow\endcsname{\let\PY@bf=\textbf\def\PY@tc##1{\textcolor[rgb]{0.67,0.13,1.00}{##1}}}
\expandafter\def\csname PY@tok@nb\endcsname{\def\PY@tc##1{\textcolor[rgb]{0.00,0.50,0.00}{##1}}}
\expandafter\def\csname PY@tok@vc\endcsname{\def\PY@tc##1{\textcolor[rgb]{0.10,0.09,0.49}{##1}}}
\expandafter\def\csname PY@tok@nt\endcsname{\let\PY@bf=\textbf\def\PY@tc##1{\textcolor[rgb]{0.00,0.50,0.00}{##1}}}
\expandafter\def\csname PY@tok@sx\endcsname{\def\PY@tc##1{\textcolor[rgb]{0.00,0.50,0.00}{##1}}}
\expandafter\def\csname PY@tok@nl\endcsname{\def\PY@tc##1{\textcolor[rgb]{0.63,0.63,0.00}{##1}}}
\expandafter\def\csname PY@tok@sr\endcsname{\def\PY@tc##1{\textcolor[rgb]{0.73,0.40,0.53}{##1}}}
\expandafter\def\csname PY@tok@cm\endcsname{\let\PY@it=\textit\def\PY@tc##1{\textcolor[rgb]{0.25,0.50,0.50}{##1}}}
\expandafter\def\csname PY@tok@w\endcsname{\def\PY@tc##1{\textcolor[rgb]{0.73,0.73,0.73}{##1}}}
\expandafter\def\csname PY@tok@kp\endcsname{\def\PY@tc##1{\textcolor[rgb]{0.00,0.50,0.00}{##1}}}
\expandafter\def\csname PY@tok@nn\endcsname{\let\PY@bf=\textbf\def\PY@tc##1{\textcolor[rgb]{0.00,0.00,1.00}{##1}}}
\expandafter\def\csname PY@tok@go\endcsname{\def\PY@tc##1{\textcolor[rgb]{0.53,0.53,0.53}{##1}}}
\expandafter\def\csname PY@tok@nd\endcsname{\def\PY@tc##1{\textcolor[rgb]{0.67,0.13,1.00}{##1}}}
\expandafter\def\csname PY@tok@mb\endcsname{\def\PY@tc##1{\textcolor[rgb]{0.40,0.40,0.40}{##1}}}
\expandafter\def\csname PY@tok@s1\endcsname{\def\PY@tc##1{\textcolor[rgb]{0.73,0.13,0.13}{##1}}}
\expandafter\def\csname PY@tok@sd\endcsname{\let\PY@it=\textit\def\PY@tc##1{\textcolor[rgb]{0.73,0.13,0.13}{##1}}}
\expandafter\def\csname PY@tok@err\endcsname{\def\PY@bc##1{\setlength{\fboxsep}{0pt}\fcolorbox[rgb]{1.00,0.00,0.00}{1,1,1}{\strut ##1}}}
\expandafter\def\csname PY@tok@k\endcsname{\let\PY@bf=\textbf\def\PY@tc##1{\textcolor[rgb]{0.00,0.50,0.00}{##1}}}
\expandafter\def\csname PY@tok@m\endcsname{\def\PY@tc##1{\textcolor[rgb]{0.40,0.40,0.40}{##1}}}
\expandafter\def\csname PY@tok@no\endcsname{\def\PY@tc##1{\textcolor[rgb]{0.53,0.00,0.00}{##1}}}
\expandafter\def\csname PY@tok@vi\endcsname{\def\PY@tc##1{\textcolor[rgb]{0.10,0.09,0.49}{##1}}}
\expandafter\def\csname PY@tok@ss\endcsname{\def\PY@tc##1{\textcolor[rgb]{0.10,0.09,0.49}{##1}}}
\expandafter\def\csname PY@tok@si\endcsname{\let\PY@bf=\textbf\def\PY@tc##1{\textcolor[rgb]{0.73,0.40,0.53}{##1}}}
\expandafter\def\csname PY@tok@gr\endcsname{\def\PY@tc##1{\textcolor[rgb]{1.00,0.00,0.00}{##1}}}
\expandafter\def\csname PY@tok@gt\endcsname{\def\PY@tc##1{\textcolor[rgb]{0.00,0.27,0.87}{##1}}}
\expandafter\def\csname PY@tok@cp\endcsname{\def\PY@tc##1{\textcolor[rgb]{0.74,0.48,0.00}{##1}}}
\expandafter\def\csname PY@tok@nc\endcsname{\let\PY@bf=\textbf\def\PY@tc##1{\textcolor[rgb]{0.00,0.00,1.00}{##1}}}
\expandafter\def\csname PY@tok@kt\endcsname{\def\PY@tc##1{\textcolor[rgb]{0.69,0.00,0.25}{##1}}}
\expandafter\def\csname PY@tok@vg\endcsname{\def\PY@tc##1{\textcolor[rgb]{0.10,0.09,0.49}{##1}}}
\expandafter\def\csname PY@tok@mo\endcsname{\def\PY@tc##1{\textcolor[rgb]{0.40,0.40,0.40}{##1}}}
\expandafter\def\csname PY@tok@mi\endcsname{\def\PY@tc##1{\textcolor[rgb]{0.40,0.40,0.40}{##1}}}
\expandafter\def\csname PY@tok@na\endcsname{\def\PY@tc##1{\textcolor[rgb]{0.49,0.56,0.16}{##1}}}
\expandafter\def\csname PY@tok@gd\endcsname{\def\PY@tc##1{\textcolor[rgb]{0.63,0.00,0.00}{##1}}}
\expandafter\def\csname PY@tok@c\endcsname{\let\PY@it=\textit\def\PY@tc##1{\textcolor[rgb]{0.25,0.50,0.50}{##1}}}
\expandafter\def\csname PY@tok@cs\endcsname{\let\PY@it=\textit\def\PY@tc##1{\textcolor[rgb]{0.25,0.50,0.50}{##1}}}
\expandafter\def\csname PY@tok@se\endcsname{\let\PY@bf=\textbf\def\PY@tc##1{\textcolor[rgb]{0.73,0.40,0.13}{##1}}}
\expandafter\def\csname PY@tok@il\endcsname{\def\PY@tc##1{\textcolor[rgb]{0.40,0.40,0.40}{##1}}}
\expandafter\def\csname PY@tok@kc\endcsname{\let\PY@bf=\textbf\def\PY@tc##1{\textcolor[rgb]{0.00,0.50,0.00}{##1}}}
\expandafter\def\csname PY@tok@gp\endcsname{\let\PY@bf=\textbf\def\PY@tc##1{\textcolor[rgb]{0.00,0.00,0.50}{##1}}}
\expandafter\def\csname PY@tok@kr\endcsname{\let\PY@bf=\textbf\def\PY@tc##1{\textcolor[rgb]{0.00,0.50,0.00}{##1}}}
\expandafter\def\csname PY@tok@mh\endcsname{\def\PY@tc##1{\textcolor[rgb]{0.40,0.40,0.40}{##1}}}
\expandafter\def\csname PY@tok@s\endcsname{\def\PY@tc##1{\textcolor[rgb]{0.73,0.13,0.13}{##1}}}
\expandafter\def\csname PY@tok@sh\endcsname{\def\PY@tc##1{\textcolor[rgb]{0.73,0.13,0.13}{##1}}}
\expandafter\def\csname PY@tok@gu\endcsname{\let\PY@bf=\textbf\def\PY@tc##1{\textcolor[rgb]{0.50,0.00,0.50}{##1}}}
\expandafter\def\csname PY@tok@ni\endcsname{\let\PY@bf=\textbf\def\PY@tc##1{\textcolor[rgb]{0.60,0.60,0.60}{##1}}}
\expandafter\def\csname PY@tok@kd\endcsname{\let\PY@bf=\textbf\def\PY@tc##1{\textcolor[rgb]{0.00,0.50,0.00}{##1}}}
\expandafter\def\csname PY@tok@cpf\endcsname{\let\PY@it=\textit\def\PY@tc##1{\textcolor[rgb]{0.25,0.50,0.50}{##1}}}
\expandafter\def\csname PY@tok@gi\endcsname{\def\PY@tc##1{\textcolor[rgb]{0.00,0.63,0.00}{##1}}}
\expandafter\def\csname PY@tok@mf\endcsname{\def\PY@tc##1{\textcolor[rgb]{0.40,0.40,0.40}{##1}}}
\expandafter\def\csname PY@tok@gh\endcsname{\let\PY@bf=\textbf\def\PY@tc##1{\textcolor[rgb]{0.00,0.00,0.50}{##1}}}
\expandafter\def\csname PY@tok@sb\endcsname{\def\PY@tc##1{\textcolor[rgb]{0.73,0.13,0.13}{##1}}}
\expandafter\def\csname PY@tok@ne\endcsname{\let\PY@bf=\textbf\def\PY@tc##1{\textcolor[rgb]{0.82,0.25,0.23}{##1}}}
\expandafter\def\csname PY@tok@o\endcsname{\def\PY@tc##1{\textcolor[rgb]{0.40,0.40,0.40}{##1}}}
\expandafter\def\csname PY@tok@sc\endcsname{\def\PY@tc##1{\textcolor[rgb]{0.73,0.13,0.13}{##1}}}
\expandafter\def\csname PY@tok@ch\endcsname{\let\PY@it=\textit\def\PY@tc##1{\textcolor[rgb]{0.25,0.50,0.50}{##1}}}
\expandafter\def\csname PY@tok@c1\endcsname{\let\PY@it=\textit\def\PY@tc##1{\textcolor[rgb]{0.25,0.50,0.50}{##1}}}
\expandafter\def\csname PY@tok@s2\endcsname{\def\PY@tc##1{\textcolor[rgb]{0.73,0.13,0.13}{##1}}}
\expandafter\def\csname PY@tok@nv\endcsname{\def\PY@tc##1{\textcolor[rgb]{0.10,0.09,0.49}{##1}}}
\expandafter\def\csname PY@tok@kn\endcsname{\let\PY@bf=\textbf\def\PY@tc##1{\textcolor[rgb]{0.00,0.50,0.00}{##1}}}

\def\PYZbs{\char`\\}
\def\PYZus{\char`\_}
\def\PYZob{\char`\{}
\def\PYZcb{\char`\}}
\def\PYZca{\char`\^}
\def\PYZam{\char`\&}
\def\PYZlt{\char`\<}
\def\PYZgt{\char`\>}
\def\PYZsh{\char`\#}
\def\PYZpc{\char`\%}
\def\PYZdl{\char`\$}
\def\PYZhy{\char`\-}
\def\PYZsq{\char`\'}
\def\PYZdq{\char`\"}
\def\PYZti{\char`\~}
% for compatibility with earlier versions
\def\PYZat{@}
\def\PYZlb{[}
\def\PYZrb{]}
\makeatother


    % Exact colors from NB
    \definecolor{incolor}{rgb}{0.0, 0.0, 0.5}
    \definecolor{outcolor}{rgb}{0.545, 0.0, 0.0}



    
    % Prevent overflowing lines due to hard-to-break entities
    \sloppy 
    % Setup hyperref package
    \hypersetup{
      breaklinks=true,  % so long urls are correctly broken across lines
      colorlinks=true,
      urlcolor=urlcolor,
      linkcolor=linkcolor,
      citecolor=citecolor,
      }
    % Slightly bigger margins than the latex defaults
    
    \geometry{verbose,tmargin=1in,bmargin=1in,lmargin=1in,rmargin=1in}
    
    

    \begin{document}
    
    
    \maketitle
    
    

    
    \section{Filtro da Média: implementação e
discussões}\label{filtro-da-muxe9dia-implementauxe7uxe3o-e-discussuxf5es}

    \subsection{Implementação (Código
fonte)}\label{implementauxe7uxe3o-cuxf3digo-fonte}

    Importação dos pacotes utilizados para simulação:

    \begin{Verbatim}[commandchars=\\\{\}]
{\color{incolor}In [{\color{incolor}1}]:} \PY{k+kn}{import} \PY{n+nn}{cv2}
        \PY{k+kn}{import} \PY{n+nn}{numpy} \PY{k}{as} \PY{n+nn}{np}
        \PY{k+kn}{import} \PY{n+nn}{matplotlib}\PY{n+nn}{.}\PY{n+nn}{pyplot} \PY{k}{as} \PY{n+nn}{plt}
        \PY{k+kn}{import} \PY{n+nn}{math} \PY{k}{as} \PY{n+nn}{m}
        \PY{k+kn}{import} \PY{n+nn}{navFunc} \PY{k}{as} \PY{n+nn}{nf}
        \PY{k+kn}{from} \PY{n+nn}{navFunc}\PY{n+nn}{.}\PY{n+nn}{cls} \PY{k}{import} \PY{n}{cls}
\end{Verbatim}

    Carregar imagem utilizando a função do OpenCV:

    \begin{Verbatim}[commandchars=\\\{\}]
{\color{incolor}In [{\color{incolor}2}]:} \PY{n}{img} \PY{o}{=} \PY{n}{cv2}\PY{o}{.}\PY{n}{imread}\PY{p}{(}\PY{l+s+s1}{\PYZsq{}}\PY{l+s+s1}{lena.png}\PY{l+s+s1}{\PYZsq{}}\PY{p}{,}\PY{n}{cv2}\PY{o}{.}\PY{n}{IMREAD\PYZus{}GRAYSCALE}\PY{p}{)}
\end{Verbatim}

    \subsubsection{Definições
preliminares:}\label{definiuxe7uxf5es-preliminares}

\begin{itemize}
\tightlist
\item
  Definir tamanho do kernel utilizado
\item
  Definir número de aplicações do filtro
\end{itemize}

    \begin{Verbatim}[commandchars=\\\{\}]
{\color{incolor}In [{\color{incolor}3}]:} \PY{c+c1}{\PYZsh{} Cria variavel do tipo struct (similar ao matlab):}
        
        \PY{n}{Filter} \PY{o}{=} \PY{n}{nf}\PY{o}{.}\PY{n}{structtype}\PY{p}{(}\PY{p}{)}                
        
        \PY{n}{Filter}\PY{o}{.}\PY{n}{img} \PY{o}{=} \PY{n}{np}\PY{o}{.}\PY{n}{array}\PY{p}{(}\PY{n}{img}\PY{p}{)}
        
        \PY{n}{Filter}\PY{o}{.}\PY{n}{imgSize} \PY{o}{=} \PY{n}{nf}\PY{o}{.}\PY{n}{structtype}\PY{p}{(}\PY{p}{)}
        \PY{n}{Filter}\PY{o}{.}\PY{n}{imgSize}\PY{o}{.}\PY{n}{lin}\PY{p}{,} \PY{n}{Filter}\PY{o}{.}\PY{n}{imgSize}\PY{o}{.}\PY{n}{col} \PY{o}{=} \PY{n}{Filter}\PY{o}{.}\PY{n}{img}\PY{o}{.}\PY{n}{shape}
        
        \PY{c+c1}{\PYZsh{}\PYZsh{}\PYZsh{}\PYZsh{}\PYZsh{}\PYZsh{}\PYZsh{}\PYZsh{}\PYZsh{}\PYZsh{}\PYZsh{}\PYZsh{}\PYZsh{}\PYZsh{}\PYZsh{}\PYZsh{}\PYZsh{}\PYZsh{}\PYZsh{}\PYZsh{} Filtro da média}
        \PY{c+c1}{\PYZsh{} Kernel def:}
        
        \PY{n}{Filter}\PY{o}{.}\PY{n}{kernelSize} \PY{o}{=} \PY{l+m+mi}{3}
        
        \PY{c+c1}{\PYZsh{} Número de aplicações do filtro}
        \PY{n}{numAp} \PY{o}{=} \PY{l+m+mi}{1}\PY{p}{;}
        
        \PY{c+c1}{\PYZsh{} Variável auxiliar para guardar a saída}
        \PY{n}{U} \PY{o}{=} \PY{n}{np}\PY{o}{.}\PY{n}{zeros}\PY{p}{(}\PY{p}{(}\PY{n}{numAp}\PY{p}{,} \PY{n}{Filter}\PY{o}{.}\PY{n}{imgSize}\PY{o}{.}\PY{n}{lin}\PY{p}{,} \PY{n}{Filter}\PY{o}{.}\PY{n}{imgSize}\PY{o}{.}\PY{n}{col}\PY{p}{)}\PY{p}{)}
\end{Verbatim}

    \subsubsection{Aplicação efetida do
método:}\label{aplicauxe7uxe3o-efetida-do-muxe9todo}

    \begin{Verbatim}[commandchars=\\\{\}]
{\color{incolor}In [{\color{incolor}4}]:} \PY{c+c1}{\PYZsh{}\PYZsh{}\PYZsh{}\PYZsh{}\PYZsh{}\PYZsh{}\PYZsh{}\PYZsh{}\PYZsh{}\PYZsh{}\PYZsh{}\PYZsh{}\PYZsh{}\PYZsh{}\PYZsh{}\PYZsh{}\PYZsh{}\PYZsh{}\PYZsh{}\PYZsh{}\PYZsh{}\PYZsh{}\PYZsh{}\PYZsh{}\PYZsh{}\PYZsh{}\PYZsh{}\PYZsh{}\PYZsh{}\PYZsh{}\PYZsh{}\PYZsh{}\PYZsh{}\PYZsh{}\PYZsh{}\PYZsh{}\PYZsh{}\PYZsh{}\PYZsh{}\PYZsh{}\PYZsh{}\PYZsh{}\PYZsh{}\PYZsh{}\PYZsh{}\PYZsh{}\PYZsh{}\PYZsh{}\PYZsh{}\PYZsh{}\PYZsh{}\PYZsh{}\PYZsh{}\PYZsh{}\PYZsh{}\PYZsh{}\PYZsh{}\PYZsh{}\PYZsh{}\PYZsh{}\PYZsh{}\PYZsh{}\PYZsh{}\PYZsh{}\PYZsh{}\PYZsh{}\PYZsh{}\PYZsh{}\PYZsh{}\PYZsh{}\PYZsh{}\PYZsh{}\PYZsh{}\PYZsh{}\PYZsh{}\PYZsh{}\PYZsh{}\PYZsh{}\PYZsh{}\PYZsh{}\PYZsh{}\PYZsh{}\PYZsh{}\PYZsh{}\PYZsh{}\PYZsh{}\PYZsh{}\PYZsh{}\PYZsh{}\PYZsh{}\PYZsh{}\PYZsh{}\PYZsh{}}
        \PY{c+c1}{\PYZsh{}\PYZsh{}\PYZsh{}\PYZsh{}\PYZsh{}\PYZsh{}\PYZsh{}\PYZsh{}\PYZsh{}\PYZsh{}\PYZsh{} Method apllication:}
        
        \PY{k}{for} \PY{n}{k} \PY{o+ow}{in} \PY{n+nb}{range}\PY{p}{(}\PY{l+m+mi}{0}\PY{p}{,} \PY{n}{numAp}\PY{p}{)}\PY{p}{:}
            \PY{k}{if} \PY{n}{k} \PY{o}{==} \PY{l+m+mi}{0}\PY{p}{:}
                \PY{n}{U}\PY{p}{[}\PY{n}{k}\PY{p}{,}\PY{p}{:}\PY{p}{,}\PY{p}{:}\PY{p}{]} \PY{o}{=} \PY{n}{nf}\PY{o}{.}\PY{n}{filterMean}\PY{p}{(}\PY{n}{Filter}\PY{p}{)}
                \PY{n+nb}{print}\PY{p}{(}\PY{n}{U}\PY{p}{[}\PY{n}{k}\PY{p}{,} \PY{p}{:}\PY{p}{,} \PY{p}{:}\PY{p}{]}\PY{p}{)}
            \PY{k}{else}\PY{p}{:}
                \PY{n}{Filter}\PY{o}{.}\PY{n}{img} \PY{o}{=} \PY{n}{U}\PY{p}{[}\PY{n}{k}\PY{o}{\PYZhy{}}\PY{l+m+mi}{1}\PY{p}{,}\PY{p}{:}\PY{p}{,}\PY{p}{:}\PY{p}{]}
                \PY{n}{U}\PY{p}{[}\PY{n}{k}\PY{p}{,} \PY{p}{:}\PY{p}{,} \PY{p}{:}\PY{p}{]} \PY{o}{=} \PY{n}{nf}\PY{o}{.}\PY{n}{filterMean}\PY{p}{(}\PY{n}{Filter}\PY{p}{)}
                \PY{n+nb}{print}\PY{p}{(}\PY{n}{U}\PY{p}{[}\PY{n}{k}\PY{p}{,}\PY{p}{:}\PY{p}{,}\PY{p}{:}\PY{p}{]}\PY{p}{)}
\end{Verbatim}

    \begin{Verbatim}[commandchars=\\\{\}]
\#\#\#\#\#\#\#\#\#\#\#\#\#\#\#\#\#\#\#\#\#\#\#\#\#\#\#\#\#\#\#\#
Process finished
Filter have been applied
\#\#\#\#\#\#\#\#\#\#\#\#\#\#\#\#\#\#\#\#\#\#\#\#\#\#\#\#\#\#\#\#
[[  34.   50.   50. {\ldots},   29.   30.   20.]
 [  51.   75.   74. {\ldots},   43.   45.   30.]
 [  51.   76.   75. {\ldots},   44.   46.   31.]
 {\ldots}, 
 [ 154.  232.  232. {\ldots},  103.  107.   73.]
 [ 152.  229.  230. {\ldots},  103.  106.   72.]
 [ 101.  152.  153. {\ldots},   68.   70.   47.]]

    \end{Verbatim}

    \subsubsection{Exibir resultados:}\label{exibir-resultados}

    \begin{itemize}
\tightlist
\item
  Imagem original:
\end{itemize}

    \begin{Verbatim}[commandchars=\\\{\}]
{\color{incolor}In [{\color{incolor}5}]:} \PY{c+c1}{\PYZsh{}\PYZsh{}\PYZsh{}\PYZsh{}\PYZsh{}\PYZsh{}\PYZsh{}\PYZsh{}\PYZsh{} Using matplotlib \PYZsh{}\PYZsh{}\PYZsh{}\PYZsh{}\PYZsh{}\PYZsh{}\PYZsh{}\PYZsh{}\PYZsh{}\PYZsh{}\PYZsh{}\PYZsh{}\PYZsh{}\PYZsh{}\PYZsh{}\PYZsh{}\PYZsh{}}
        \PY{n}{plt}\PY{o}{.}\PY{n}{figure}\PY{p}{(}\PY{l+m+mi}{1}\PY{p}{)}
        \PY{n}{plt}\PY{o}{.}\PY{n}{imshow}\PY{p}{(}\PY{n}{img}\PY{p}{,} \PY{l+s+s1}{\PYZsq{}}\PY{l+s+s1}{gray}\PY{l+s+s1}{\PYZsq{}}\PY{p}{)}
        \PY{n}{plt}\PY{o}{.}\PY{n}{title}\PY{p}{(}\PY{l+s+s1}{\PYZsq{}}\PY{l+s+s1}{Imagem Original}\PY{l+s+s1}{\PYZsq{}}\PY{p}{)}
        \PY{n}{plt}\PY{o}{.}\PY{n}{show}\PY{p}{(}\PY{p}{)}
\end{Verbatim}

    \begin{center}
    \adjustimage{max size={0.9\linewidth}{0.9\paperheight}}{filtroMediaTeste01-ipt_files/filtroMediaTeste01-ipt_12_0.png}
    \end{center}
    { \hspace*{\fill} \\}
    
    \begin{Verbatim}[commandchars=\\\{\}]
{\color{incolor}In [{\color{incolor}6}]:} \PY{n}{plt}\PY{o}{.}\PY{n}{figure}\PY{p}{(}\PY{l+m+mi}{2}\PY{p}{)}
        \PY{n}{plt}\PY{o}{.}\PY{n}{imshow}\PY{p}{(}\PY{n}{U}\PY{p}{[}\PY{p}{(}\PY{n}{numAp} \PY{o}{\PYZhy{}} \PY{l+m+mi}{1}\PY{p}{)}\PY{p}{,}\PY{p}{:}\PY{p}{,}\PY{p}{:}\PY{p}{]}\PY{p}{,} \PY{l+s+s1}{\PYZsq{}}\PY{l+s+s1}{gray}\PY{l+s+s1}{\PYZsq{}}\PY{p}{)}
        \PY{n}{plt}\PY{o}{.}\PY{n}{title}\PY{p}{(}\PY{l+s+s1}{\PYZsq{}}\PY{l+s+s1}{Imagem c/ aplicação do filtro da média }\PY{l+s+si}{\PYZpc{}d}\PY{l+s+s1}{ vez}\PY{l+s+s1}{\PYZsq{}} \PY{o}{\PYZpc{}}\PY{k}{numAp})
        \PY{n}{plt}\PY{o}{.}\PY{n}{show}\PY{p}{(}\PY{p}{)}
\end{Verbatim}

    \begin{center}
    \adjustimage{max size={0.9\linewidth}{0.9\paperheight}}{filtroMediaTeste01-ipt_files/filtroMediaTeste01-ipt_13_0.png}
    \end{center}
    { \hspace*{\fill} \\}
    
    \paragraph{Apêndice 01 - Função para cálculo do filtro da
média:}\label{apuxeandice-01---funuxe7uxe3o-para-cuxe1lculo-do-filtro-da-muxe9dia}

    \begin{Verbatim}[commandchars=\\\{\}]
{\color{incolor}In [{\color{incolor}7}]:} \PY{k}{def} \PY{n+nf}{filterMean} \PY{p}{(}\PY{n}{Filter}\PY{p}{)}\PY{p}{:}
            \PY{c+c1}{\PYZsh{}\PYZsh{}\PYZsh{} Imports}
            \PY{k+kn}{import} \PY{n+nn}{numpy} \PY{k}{as} \PY{n+nn}{np}
            \PY{k+kn}{import} \PY{n+nn}{matplotlib}\PY{n+nn}{.}\PY{n+nn}{pyplot} \PY{k}{as} \PY{n+nn}{plt}
            \PY{k+kn}{import} \PY{n+nn}{math} \PY{k}{as} \PY{n+nn}{m}
            \PY{k+kn}{import} \PY{n+nn}{navFunc} \PY{k}{as} \PY{n+nn}{nf}
        
            \PY{c+c1}{\PYZsh{} Load image into numpy matrix}
        
            \PY{n}{A} \PY{o}{=} \PY{n}{Filter}\PY{o}{.}\PY{n}{img}
        
            \PY{n}{size} \PY{o}{=} \PY{n}{nf}\PY{o}{.}\PY{n}{structtype}\PY{p}{(}\PY{p}{)}
            \PY{n}{size}\PY{o}{.}\PY{n}{A} \PY{o}{=} \PY{n}{nf}\PY{o}{.}\PY{n}{structtype}\PY{p}{(}\PY{p}{)}
            \PY{n}{size}\PY{o}{.}\PY{n}{A}\PY{o}{.}\PY{n}{lin}\PY{p}{,} \PY{n}{size}\PY{o}{.}\PY{n}{A}\PY{o}{.}\PY{n}{col} \PY{o}{=} \PY{n}{A}\PY{o}{.}\PY{n}{shape}
        
            \PY{c+c1}{\PYZsh{}\PYZsh{}\PYZsh{}\PYZsh{}\PYZsh{}\PYZsh{}\PYZsh{}\PYZsh{}\PYZsh{}\PYZsh{}\PYZsh{}\PYZsh{}\PYZsh{}\PYZsh{}\PYZsh{}\PYZsh{}\PYZsh{}\PYZsh{}\PYZsh{}\PYZsh{} Mean filter}
            \PY{c+c1}{\PYZsh{}\PYZsh{} Pre\PYZhy{}set steps:}
            \PY{n}{Filter}\PY{o}{.}\PY{n}{kernel} \PY{o}{=} \PY{n}{np}\PY{o}{.}\PY{n}{ones}\PY{p}{(}\PY{p}{(}\PY{n}{Filter}\PY{o}{.}\PY{n}{kernelSize}\PY{p}{,} \PY{n}{Filter}\PY{o}{.}\PY{n}{kernelSize}\PY{p}{)}\PY{p}{)}
            \PY{c+c1}{\PYZsh{}\PYZsh{}\PYZsh{}\PYZsh{}\PYZsh{}\PYZsh{}\PYZsh{}\PYZsh{}\PYZsh{}\PYZsh{}\PYZsh{}\PYZsh{}\PYZsh{}\PYZsh{}\PYZsh{}\PYZsh{}\PYZsh{}}
            \PY{n}{central} \PY{o}{=} \PY{n}{m}\PY{o}{.}\PY{n}{floor}\PY{p}{(}\PY{p}{(}\PY{n}{Filter}\PY{o}{.}\PY{n}{kernelSize} \PY{o}{/} \PY{l+m+mi}{2}\PY{p}{)}\PY{p}{)}
            \PY{n}{C} \PY{o}{=} \PY{n}{np}\PY{o}{.}\PY{n}{zeros}\PY{p}{(}\PY{p}{(}\PY{n}{size}\PY{o}{.}\PY{n}{A}\PY{o}{.}\PY{n}{lin} \PY{o}{+} \PY{n}{central} \PY{o}{*} \PY{l+m+mi}{2}\PY{p}{,} \PY{n}{size}\PY{o}{.}\PY{n}{A}\PY{o}{.}\PY{n}{col} \PY{o}{+} \PY{n}{central} \PY{o}{*} \PY{l+m+mi}{2}\PY{p}{)}\PY{p}{)}
            \PY{n}{C}\PY{p}{[}\PY{p}{(}\PY{l+m+mi}{0} \PY{o}{+} \PY{n}{central}\PY{p}{)}\PY{p}{:}\PY{p}{(}\PY{n}{size}\PY{o}{.}\PY{n}{A}\PY{o}{.}\PY{n}{lin} \PY{o}{+} \PY{n}{central}\PY{p}{)}\PY{p}{,} \PY{p}{(}\PY{l+m+mi}{0} \PY{o}{+} \PY{n}{central}\PY{p}{)}\PY{p}{:}\PY{p}{(}\PY{n}{size}\PY{o}{.}\PY{n}{A}\PY{o}{.}\PY{n}{col} \PY{o}{+} \PY{n}{central}\PY{p}{)}\PY{p}{]} \PY{o}{=} \PY{n}{A}
        
            \PY{c+c1}{\PYZsh{}\PYZsh{}\PYZsh{}\PYZsh{}\PYZsh{}\PYZsh{}\PYZsh{}\PYZsh{}\PYZsh{}\PYZsh{}\PYZsh{}\PYZsh{}\PYZsh{}\PYZsh{}\PYZsh{}\PYZsh{}\PYZsh{}}
            \PY{c+c1}{\PYZsh{}\PYZsh{}  Run the kernel over the matrix (similar to convolution):}
            \PY{c+c1}{\PYZsh{}\PYZsh{}\PYZsh{}\PYZsh{}\PYZsh{}\PYZsh{}\PYZsh{}\PYZsh{}\PYZsh{}\PYZsh{}\PYZsh{}\PYZsh{}\PYZsh{}\PYZsh{}\PYZsh{}\PYZsh{}\PYZsh{}}
            \PY{n}{soma} \PY{o}{=} \PY{l+m+mi}{0}\PY{p}{;}
            \PY{n}{D} \PY{o}{=} \PY{n}{np}\PY{o}{.}\PY{n}{zeros}\PY{p}{(}\PY{n}{A}\PY{o}{.}\PY{n}{shape}\PY{p}{)}
        
            \PY{k}{for} \PY{n}{j} \PY{o+ow}{in} \PY{n+nb}{range}\PY{p}{(}\PY{p}{(}\PY{l+m+mi}{0}\PY{p}{)}\PY{p}{,} \PY{n}{size}\PY{o}{.}\PY{n}{A}\PY{o}{.}\PY{n}{lin}\PY{p}{)}\PY{p}{:}
                \PY{k}{for} \PY{n}{k} \PY{o+ow}{in} \PY{n+nb}{range}\PY{p}{(}\PY{p}{(}\PY{l+m+mi}{0}\PY{p}{)}\PY{p}{,} \PY{n}{size}\PY{o}{.}\PY{n}{A}\PY{o}{.}\PY{n}{col}\PY{p}{)}\PY{p}{:}
                    \PY{c+c1}{\PYZsh{} Run kernel in one matrix\PYZsq{}s elements}
                    \PY{k}{for} \PY{n}{kl} \PY{o+ow}{in} \PY{n+nb}{range}\PY{p}{(}\PY{l+m+mi}{0}\PY{p}{,} \PY{n}{Filter}\PY{o}{.}\PY{n}{kernelSize}\PY{p}{)}\PY{p}{:}
                        \PY{k}{for} \PY{n}{kk} \PY{o+ow}{in} \PY{n+nb}{range}\PY{p}{(}\PY{l+m+mi}{0}\PY{p}{,} \PY{n}{Filter}\PY{o}{.}\PY{n}{kernelSize}\PY{p}{)}\PY{p}{:}
                           
                            \PY{n}{soma} \PY{o}{=} \PY{p}{(}\PY{n}{C}\PY{p}{[}\PY{n}{j} \PY{o}{+} \PY{n}{kl}\PY{p}{,} \PY{n}{k} \PY{o}{+} \PY{n}{kk}\PY{p}{]} \PY{o}{*} \PY{n}{Filter}\PY{o}{.}\PY{n}{kernel}\PY{p}{[}\PY{n}{kl}\PY{p}{,} \PY{n}{kk}\PY{p}{]}\PY{p}{)} \PY{o}{+} \PY{n}{soma}
        
                    \PY{n}{value} \PY{o}{=} \PY{n}{m}\PY{o}{.}\PY{n}{ceil}\PY{p}{(}\PY{p}{(}\PY{n}{soma} \PY{o}{/} \PY{p}{(}\PY{n}{Filter}\PY{o}{.}\PY{n}{kernelSize} \PY{o}{*} \PY{n}{Filter}\PY{o}{.}\PY{n}{kernelSize}\PY{p}{)}\PY{p}{)}\PY{p}{)}
                    \PY{n}{soma} \PY{o}{=} \PY{l+m+mi}{0}
                    \PY{n}{D}\PY{p}{[}\PY{n}{j}\PY{p}{,} \PY{n}{k}\PY{p}{]} \PY{o}{=} \PY{n}{value}
        
            \PY{n}{D} \PY{o}{=} \PY{n}{np}\PY{o}{.}\PY{n}{uint8}\PY{p}{(}\PY{n}{D}\PY{p}{)}
        
            \PY{n+nb}{print}\PY{p}{(}\PY{l+s+s1}{\PYZsq{}}\PY{l+s+s1}{\PYZsh{}\PYZsh{}\PYZsh{}\PYZsh{}\PYZsh{}\PYZsh{}\PYZsh{}\PYZsh{}\PYZsh{}\PYZsh{}\PYZsh{}\PYZsh{}\PYZsh{}\PYZsh{}\PYZsh{}\PYZsh{}\PYZsh{}\PYZsh{}\PYZsh{}\PYZsh{}\PYZsh{}\PYZsh{}\PYZsh{}\PYZsh{}\PYZsh{}\PYZsh{}\PYZsh{}\PYZsh{}\PYZsh{}\PYZsh{}\PYZsh{}\PYZsh{}}\PY{l+s+s1}{\PYZsq{}}\PY{p}{)}
            \PY{n+nb}{print}\PY{p}{(}\PY{l+s+s1}{\PYZsq{}}\PY{l+s+s1}{Process finished}\PY{l+s+s1}{\PYZsq{}}\PY{p}{)}
            \PY{n+nb}{print}\PY{p}{(}\PY{l+s+s1}{\PYZsq{}}\PY{l+s+s1}{Filter have been applied}\PY{l+s+s1}{\PYZsq{}}\PY{p}{)}
            \PY{n+nb}{print}\PY{p}{(}\PY{l+s+s1}{\PYZsq{}}\PY{l+s+s1}{\PYZsh{}\PYZsh{}\PYZsh{}\PYZsh{}\PYZsh{}\PYZsh{}\PYZsh{}\PYZsh{}\PYZsh{}\PYZsh{}\PYZsh{}\PYZsh{}\PYZsh{}\PYZsh{}\PYZsh{}\PYZsh{}\PYZsh{}\PYZsh{}\PYZsh{}\PYZsh{}\PYZsh{}\PYZsh{}\PYZsh{}\PYZsh{}\PYZsh{}\PYZsh{}\PYZsh{}\PYZsh{}\PYZsh{}\PYZsh{}\PYZsh{}\PYZsh{}}\PY{l+s+s1}{\PYZsq{}}\PY{p}{)}
        
            \PY{k}{return} \PY{n}{D}
\end{Verbatim}

    \section{Discussões sobre o
método}\label{discussuxf5es-sobre-o-muxe9todo}

    A ideia por trás da filtro dos filtros lineares de suavização, neste
caso o filtro da média, reside na necessidade de exibir apenas partes de
interesse de uma imagem. Com intuitos explicativos faz-se primeramente
uma analogia com o processamento de um sinal digital.

No território de processamento de sinais unidimensionas a aplicação de
uma média janelada funciona como um filtro passa-baixa, atenuando
frequências maiores que a frequência de corte. O exemplo na figura a
seguir ilustra um sinal aleatório de 10 amostras a o valor médio marcado
me vemelho. Nota-se a que houve uma atenuação(filtragem) dos valores
maiores que a média, por tal razão este é um filtro passa-baixa com
frequência de corte estabelecida pelo valor médio das amostras. Este
filtro pode ser bastante útil em sinais corrompidos por ruídos(comumente
aleatórios e de alta frequência), portanto é possível a remoção do mesmo
desde que haja um número suficiente de amostras.

    \begin{Verbatim}[commandchars=\\\{\}]
{\color{incolor}In [{\color{incolor}8}]:} \PY{n}{values} \PY{o}{=} \PY{n}{np}\PY{o}{.}\PY{n}{random}\PY{o}{.}\PY{n}{randn}\PY{p}{(}\PY{l+m+mi}{10}\PY{p}{)}
        \PY{n+nb}{print}\PY{p}{(}\PY{l+s+s1}{\PYZsq{}}\PY{l+s+s1}{Valores:}\PY{l+s+s1}{\PYZsq{}}\PY{p}{)}
        \PY{n+nb}{print}\PY{p}{(}\PY{n}{values}\PY{p}{)}
        \PY{n}{avg} \PY{o}{=} \PY{n}{np}\PY{o}{.}\PY{n}{average}\PY{p}{(}\PY{n}{values}\PY{p}{)}
        \PY{n+nb}{print}\PY{p}{(}\PY{l+s+s1}{\PYZsq{}}\PY{l+s+s1}{Média:}\PY{l+s+s1}{\PYZsq{}}\PY{p}{)}
        \PY{n+nb}{print}\PY{p}{(}\PY{n}{avg}\PY{p}{)}
\end{Verbatim}

    \begin{Verbatim}[commandchars=\\\{\}]
Valores:
[-1.8214013   0.40779237 -0.38503472 -0.55850611  2.25670963 -1.91303472
 -2.43157197  2.10198381  0.10426469  0.17594206]
Média:
-0.206285624876

    \end{Verbatim}

    \begin{Verbatim}[commandchars=\\\{\}]
{\color{incolor}In [{\color{incolor}9}]:} \PY{n}{plt}\PY{o}{.}\PY{n}{figure}\PY{p}{(}\PY{l+m+mi}{3}\PY{p}{)}
        \PY{n}{plt}\PY{o}{.}\PY{n}{plot}\PY{p}{(}\PY{n}{values}\PY{p}{)}
        \PY{n}{plt}\PY{o}{.}\PY{n}{plot}\PY{p}{(}\PY{n+nb}{int}\PY{p}{(}\PY{n}{np}\PY{o}{.}\PY{n}{size}\PY{p}{(}\PY{n}{values}\PY{p}{)}\PY{o}{/}\PY{l+m+mi}{2}\PY{p}{)}\PY{p}{,} \PY{n}{avg}\PY{p}{,} \PY{l+s+s1}{\PYZsq{}}\PY{l+s+s1}{ro}\PY{l+s+s1}{\PYZsq{}}\PY{p}{,} \PY{n}{markersize}\PY{o}{=}\PY{l+m+mf}{20.0}\PY{p}{)}
        \PY{n}{plt}\PY{o}{.}\PY{n}{title}\PY{p}{(}\PY{l+s+s1}{\PYZsq{}}\PY{l+s+s1}{Sinal aleatório e valor médio de }\PY{l+s+si}{\PYZpc{}d}\PY{l+s+s1}{ amostras}\PY{l+s+s1}{\PYZsq{}} \PY{o}{\PYZpc{}}\PY{p}{(}\PY{n}{np}\PY{o}{.}\PY{n}{size}\PY{p}{(}\PY{n}{values}\PY{p}{)}\PY{p}{)}\PY{p}{)}
        \PY{n}{plt}\PY{o}{.}\PY{n}{show}\PY{p}{(}\PY{p}{)}
\end{Verbatim}

    \begin{center}
    \adjustimage{max size={0.9\linewidth}{0.9\paperheight}}{filtroMediaTeste01-ipt_files/filtroMediaTeste01-ipt_19_0.png}
    \end{center}
    { \hspace*{\fill} \\}
    
    No âmbito do processamento digital de imagens tal teoria confere ao
filtro da média a capacidade de borrar a imagem,
reduzindo(``atenuando'') transições abruptas de intensidade. A figura a
seguir exibe a saída de uma imagem submetida ao filtro, sucessivamente,
por 50 vezes. A transição entre diferenças abruptas de intensidade
tornou-se mais suave, transformando regiões originalmente bastante
distas entre preto e branco em uma escala de cinza de transição.

É importante salientar que houve uma atenuação de valores elevados(255)
deslocando regiões originalmente brancas para tons suaves de cinza. Em
especial nas bordas da imagem, reduzindo a área útil da mesma.

    \begin{Verbatim}[commandchars=\\\{\}]
{\color{incolor}In [{\color{incolor}10}]:} \PY{n}{img2} \PY{o}{=} \PY{n}{cv2}\PY{o}{.}\PY{n}{imread}\PY{p}{(}\PY{l+s+s1}{\PYZsq{}}\PY{l+s+s1}{03.png}\PY{l+s+s1}{\PYZsq{}}\PY{p}{,}\PY{n}{cv2}\PY{o}{.}\PY{n}{IMREAD\PYZus{}GRAYSCALE}\PY{p}{)}
                 
         \PY{n}{plt}\PY{o}{.}\PY{n}{figure}\PY{p}{(}\PY{l+m+mi}{3}\PY{o}{+}\PY{l+m+mi}{1}\PY{p}{)}
         \PY{n}{plt}\PY{o}{.}\PY{n}{imshow}\PY{p}{(}\PY{n}{img2}\PY{p}{,} \PY{l+s+s1}{\PYZsq{}}\PY{l+s+s1}{gray}\PY{l+s+s1}{\PYZsq{}}\PY{p}{)}
         \PY{n}{plt}\PY{o}{.}\PY{n}{show}\PY{p}{(}\PY{p}{)}
\end{Verbatim}

    \begin{center}
    \adjustimage{max size={0.9\linewidth}{0.9\paperheight}}{filtroMediaTeste01-ipt_files/filtroMediaTeste01-ipt_21_0.png}
    \end{center}
    { \hspace*{\fill} \\}
    
    A escolha ideal da máscara(kernel) pode ser baseada na necessidade de
``reduzir'' detalhes não relevantes da imagem, pois o funcionamento do
filtro confere a este a capacidade de borrar imagens menores que a
tamanho da máscara. Diante do comparativo a seguir nota-se que o kernel
de tamanho 3x3 borrou os menores detalhes(cujos principais pixels eram
menores que o kernel), talvez impercetíveis a uma olhar rápido, enquanto
o kernel 17x17 de forma marcante borrou ``grandes'' detalhes da imagem
como um todo.

    \begin{Verbatim}[commandchars=\\\{\}]
{\color{incolor}In [{\color{incolor}11}]:} \PY{n}{img3} \PY{o}{=} \PY{n}{cv2}\PY{o}{.}\PY{n}{imread}\PY{p}{(}\PY{l+s+s1}{\PYZsq{}}\PY{l+s+s1}{kernels\PYZhy{}01.png}\PY{l+s+s1}{\PYZsq{}}\PY{p}{,}\PY{n}{cv2}\PY{o}{.}\PY{n}{IMREAD\PYZus{}GRAYSCALE}\PY{p}{)}
         \PY{n}{img4} \PY{o}{=} \PY{n}{cv2}\PY{o}{.}\PY{n}{imread}\PY{p}{(}\PY{l+s+s1}{\PYZsq{}}\PY{l+s+s1}{kernels\PYZhy{}02.png}\PY{l+s+s1}{\PYZsq{}}\PY{p}{,}\PY{n}{cv2}\PY{o}{.}\PY{n}{IMREAD\PYZus{}GRAYSCALE}\PY{p}{)}
         \PY{n}{img5} \PY{o}{=} \PY{n}{cv2}\PY{o}{.}\PY{n}{imread}\PY{p}{(}\PY{l+s+s1}{\PYZsq{}}\PY{l+s+s1}{kernels\PYZhy{}03.png}\PY{l+s+s1}{\PYZsq{}}\PY{p}{,}\PY{n}{cv2}\PY{o}{.}\PY{n}{IMREAD\PYZus{}GRAYSCALE}\PY{p}{)}
         \PY{n}{img6} \PY{o}{=} \PY{n}{cv2}\PY{o}{.}\PY{n}{imread}\PY{p}{(}\PY{l+s+s1}{\PYZsq{}}\PY{l+s+s1}{kernels\PYZhy{}04.png}\PY{l+s+s1}{\PYZsq{}}\PY{p}{,}\PY{n}{cv2}\PY{o}{.}\PY{n}{IMREAD\PYZus{}GRAYSCALE}\PY{p}{)}
         \PY{n}{img7} \PY{o}{=} \PY{n}{cv2}\PY{o}{.}\PY{n}{imread}\PY{p}{(}\PY{l+s+s1}{\PYZsq{}}\PY{l+s+s1}{kernels\PYZhy{}05.png}\PY{l+s+s1}{\PYZsq{}}\PY{p}{,}\PY{n}{cv2}\PY{o}{.}\PY{n}{IMREAD\PYZus{}GRAYSCALE}\PY{p}{)}
         \PY{n}{img8} \PY{o}{=} \PY{n}{cv2}\PY{o}{.}\PY{n}{imread}\PY{p}{(}\PY{l+s+s1}{\PYZsq{}}\PY{l+s+s1}{kernels\PYZhy{}06.png}\PY{l+s+s1}{\PYZsq{}}\PY{p}{,}\PY{n}{cv2}\PY{o}{.}\PY{n}{IMREAD\PYZus{}GRAYSCALE}\PY{p}{)}
         \PY{n}{img9} \PY{o}{=} \PY{n}{cv2}\PY{o}{.}\PY{n}{imread}\PY{p}{(}\PY{l+s+s1}{\PYZsq{}}\PY{l+s+s1}{kernels\PYZhy{}07.png}\PY{l+s+s1}{\PYZsq{}}\PY{p}{,}\PY{n}{cv2}\PY{o}{.}\PY{n}{IMREAD\PYZus{}GRAYSCALE}\PY{p}{)}
         
                
         \PY{n}{plt}\PY{o}{.}\PY{n}{figure}\PY{p}{(}\PY{l+m+mi}{1}\PY{p}{)}
         \PY{n}{plt}\PY{o}{.}\PY{n}{imshow}\PY{p}{(}\PY{n}{img3}\PY{p}{,} \PY{l+s+s1}{\PYZsq{}}\PY{l+s+s1}{gray}\PY{l+s+s1}{\PYZsq{}}\PY{p}{)}
         
         \PY{n}{plt}\PY{o}{.}\PY{n}{figure}\PY{p}{(}\PY{l+m+mi}{2}\PY{p}{)}
         \PY{n}{plt}\PY{o}{.}\PY{n}{imshow}\PY{p}{(}\PY{n}{img4}\PY{p}{,} \PY{l+s+s1}{\PYZsq{}}\PY{l+s+s1}{gray}\PY{l+s+s1}{\PYZsq{}}\PY{p}{)}
         
         \PY{n}{plt}\PY{o}{.}\PY{n}{figure}\PY{p}{(}\PY{l+m+mi}{3}\PY{p}{)}
         \PY{n}{plt}\PY{o}{.}\PY{n}{imshow}\PY{p}{(}\PY{n}{img5}\PY{p}{,} \PY{l+s+s1}{\PYZsq{}}\PY{l+s+s1}{gray}\PY{l+s+s1}{\PYZsq{}}\PY{p}{)}
         
         \PY{n}{plt}\PY{o}{.}\PY{n}{figure}\PY{p}{(}\PY{l+m+mi}{1}\PY{p}{)}
         \PY{n}{plt}\PY{o}{.}\PY{n}{imshow}\PY{p}{(}\PY{n}{img6}\PY{p}{,} \PY{l+s+s1}{\PYZsq{}}\PY{l+s+s1}{gray}\PY{l+s+s1}{\PYZsq{}}\PY{p}{)}
         
         \PY{n}{plt}\PY{o}{.}\PY{n}{figure}\PY{p}{(}\PY{l+m+mi}{5}\PY{p}{)}
         \PY{n}{plt}\PY{o}{.}\PY{n}{imshow}\PY{p}{(}\PY{n}{img7}\PY{p}{,} \PY{l+s+s1}{\PYZsq{}}\PY{l+s+s1}{gray}\PY{l+s+s1}{\PYZsq{}}\PY{p}{)}
         
         \PY{n}{plt}\PY{o}{.}\PY{n}{figure}\PY{p}{(}\PY{l+m+mi}{6}\PY{p}{)}
         \PY{n}{plt}\PY{o}{.}\PY{n}{imshow}\PY{p}{(}\PY{n}{img8}\PY{p}{,} \PY{l+s+s1}{\PYZsq{}}\PY{l+s+s1}{gray}\PY{l+s+s1}{\PYZsq{}}\PY{p}{)}
         
         \PY{n}{plt}\PY{o}{.}\PY{n}{figure}\PY{p}{(}\PY{l+m+mi}{7}\PY{p}{)}
         \PY{n}{plt}\PY{o}{.}\PY{n}{imshow}\PY{p}{(}\PY{n}{img9}\PY{p}{,} \PY{l+s+s1}{\PYZsq{}}\PY{l+s+s1}{gray}\PY{l+s+s1}{\PYZsq{}}\PY{p}{)}
         
         \PY{n}{plt}\PY{o}{.}\PY{n}{show}\PY{p}{(}\PY{p}{)} 
\end{Verbatim}

    \begin{center}
    \adjustimage{max size={0.9\linewidth}{0.9\paperheight}}{filtroMediaTeste01-ipt_files/filtroMediaTeste01-ipt_23_0.png}
    \end{center}
    { \hspace*{\fill} \\}
    
    \begin{center}
    \adjustimage{max size={0.9\linewidth}{0.9\paperheight}}{filtroMediaTeste01-ipt_files/filtroMediaTeste01-ipt_23_1.png}
    \end{center}
    { \hspace*{\fill} \\}
    
    \begin{center}
    \adjustimage{max size={0.9\linewidth}{0.9\paperheight}}{filtroMediaTeste01-ipt_files/filtroMediaTeste01-ipt_23_2.png}
    \end{center}
    { \hspace*{\fill} \\}
    
    \begin{center}
    \adjustimage{max size={0.9\linewidth}{0.9\paperheight}}{filtroMediaTeste01-ipt_files/filtroMediaTeste01-ipt_23_3.png}
    \end{center}
    { \hspace*{\fill} \\}
    
    \begin{center}
    \adjustimage{max size={0.9\linewidth}{0.9\paperheight}}{filtroMediaTeste01-ipt_files/filtroMediaTeste01-ipt_23_4.png}
    \end{center}
    { \hspace*{\fill} \\}
    
    \begin{center}
    \adjustimage{max size={0.9\linewidth}{0.9\paperheight}}{filtroMediaTeste01-ipt_files/filtroMediaTeste01-ipt_23_5.png}
    \end{center}
    { \hspace*{\fill} \\}
    
    

    % Add a bibliography block to the postdoc
    
    
    
    \end{document}
